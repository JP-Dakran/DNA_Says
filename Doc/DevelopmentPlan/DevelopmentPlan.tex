\documentclass{article}
\usepackage{hyperref}
\usepackage{booktabs}
\usepackage{indentfirst}
\hypersetup{colorlinks=true,
    linkcolor=blue,
    citecolor=blue,
    filecolor=blue,
    urlcolor=blue,
    unicode=false}
\usepackage{tabularx}
\usepackage{url}
\usepackage{hyperref}
\hypersetup{colorlinks=true,
    linkcolor=blue,
    citecolor=blue,
    filecolor=blue,
    urlcolor=blue,
    unicode=false}
\title{SE 3XA3: Development Plan\\DNA Says}
\author{Team 10, Team Name: DNA
		\\ Kareem Abdel Mesih (abdelk2)
		\\ John-Paul Dakran (dakranj)
		\\ Shady Nessim (nessimss)
}

\date{}
\begin{document}
\maketitle
\newpage
\tableofcontents

\begin{table}[hp]
\caption{Revision History} \label{TblRevisionHistory}
\begin{tabularx}{\textwidth}{llX}
\toprule
\textbf{Date} & \textbf{Developer(s)} & \textbf{Change}\\
\midrule
2016/09/29 & Shady Nessim & Rough Draft\\
2016/09/29 & John-Paul Dakran &  Revision 0\\
2016/09/29 & Kareem Abdel Mesih &  Revision 0.1\\
2016/09/30 & Shady Nessim & Linked Gantt Chart\\
2016/09/30 & All & Revision 0.2\\
2016/10/10 & John-Paul Dakran & Revision 0.3\\
2016/10/10 & Kareem Abdel Mesih & Revision 0.4\\
\bottomrule
\end{tabularx}
\end{table}
\newpage
\newpage


\section{Team Meeting Plan} 
The team will meet at a minimum once a week and as the workload increases, team meetings will increase in frequency as agreed upon by the team. The weekly team meeting will occur every Wednesday at 10:30 AM - 11:20 AM in the ETB Cafe.\\
\\
\indent Kareem will chair the weekly meetings. The chair's duty is to organize each meeting. He will analyze the different sections of the weekly deliverables and will therefore cover all requirements. He will also prepare discussions that are relevant to the work that needs to be completed.\\
\\
\indent The meetings will begin with a discussion about the previous week's deliverable to decide upon any changes that need to be made and to evaluate group decisions. Once complete, the meeting will move towards discussing the current week's deliverable. Each group member will express their opinion on the work that needs to be completed. Finally, the work will be distributed evenly among the group members and the meeting will commence.\\
\\
\indent During the  meetings, a set of notes or "Meeting Minutes" will be recorded to have a record of what occurred during each meeting. This will provide a clear history of what was discussed, decided upon, and a list each person's duties. Group members will alternate recording the meeting minutes - beginning with Kareem, then John-Paul, and finally Shady; then we will repeat.


\section{Team Communication Plan}
The Git Repository will serve as a means for group members and teaching staff to share and view project files. The files that are pushed to Git will be committed with a meaningful message describing the changes that have been made to the specific file. \\
\\
\indent Email and Facebook Messenger will be the platforms that the group members use to communicate with each other. They will serve the purpose of discussing ideas, sharing files, and confirming team meetings. Each group member should regurarly check their email and Facebook Messenger account to stay updated with ongoing team activities.\\
\\
\indent All group members will exchange telephone numbers and will be used to contact each other to discuss ideas and plan team meetings. Any questions that need to be answered as soon as possible should be asked through SMS messages for fastest response time. \\
\\
\indent All issues that occur will be posted on the Issue Tracker. When an issue occurs, all group members will have a discussion about the issue and pitch ideas on how to solve the issue. A meeting will be called if the resolution needs to be discussed further. Once an issue is solved, it will be noted on the Issue Tracker.

\section{Team Member Roles}
The team leader will be Kareem. Kareem will also chair each of the meetings. Group members will alternate recording the meeting minutes - beginning with Kareem, then John-Paul, and finally Shady; then we will repeat.

\subsection{Knowledge and Experience}

All values below are a range between 0 (no experience) - 7 (expert experience)

\begin{itemize}
\item Kareem
\begin{itemize}
\item Documentation: 7
\item Git: 2
\item LaTeX: 3
\item Technology: 7
\end{itemize}

\item  JP
\begin{itemize}
\item Documentation: 5 
\item Git: 5 
\item LaTeX: 3 
\item Technology: 6
\end{itemize}

\item Shady
\begin{itemize}
\item Documentation: 5 
\item Git: 4 
\item LaTeX: 2 
\item Technology: 5
\end{itemize}
\end{itemize}

* Technology refers to Python \& Pygame.\\
\\
\indent During the process of developing this project, each team member will get a chance to explore all aspects of this project in order to further develop their understanding of the elements listed above. This should allow each member to understand and excel in all given aspects. 


\subsection{Roles}

\paragraph{Kareem\\}

Will be in charge or producing all required soundclips. He will create all neccessary buttons and simple graphics.
His main duty is the program logic and the back-end of this project. He will also prepare meeting discussions that depend on the analysis of the weekly deliverables.

\paragraph{John-Paul\\}
Will be in charge of all LaTeX preparations and organization required for documenting the aspects of this project.
He will also program the user interface and the front-end of this project.

\paragraph{Shady\\}
Will be in charge of all Git preparations and organization of this project. He will also test the code using various known and effective
methods. He will be maintaining the code.

\section{Git Workflow Plan}

All issues will be posted on the Issue Tracker to be discussed and resolved. In addition, descriptive commit messages will be used when committing to Git to reflect the changes that have been applied. There will be milestones set, along with their expected completion date and posted on Git for all group members to be aware of. Lastly, labels will be used to prioritize and organize issues, and merge requests when needed.

\section{Proof of Concept Demonstration Plan}

The implementation of this project should not be difficult, however the process of recording the sounds will be time consuming. As for testing, it also should not be difficult, as strings will be used and compared to test the addition of new moves to the previous pattern. There will be testing with family and friends that will play an integral role in determining the functionality of the program. Furthermore, this code will run on any platform that can run Python. Pygame is required to develop this project. It is simple to install once one knows which versions are compatible with the current version of Python (3.5).\\
\\
\indent There will be at least one mode implemented before the demonstration date and that is to prove that the team is able to use the same concepts used to build that mode to create two more in the span of the project.

\paragraph{Overcoming Risks:}
With regards to compatibility issues, the latest pygame was released in 2009, which fortunately enough is compatible with the current Python version available. However, if Python releases a new update, it might not be compatible anymore. The option to download an older version of Python is still available however, and can be used to continue the project. In fact, there would be no need to update, if one happens to be available.

\section{Technology}
Python will be used to develop this project. It will run in its basic IDLE Version 3.5. Framework testing will be automated to test the different cases and outcomes of the game. Family and friends will test the overall functionality and performance of the game. LaTeX will be used to generate required documents.

\section{Coding Style}
Descriptive variable names are required to reflect what each variable holds. Also, descriptive method names to aid in understanding what each method is doing. There will be consistency in spacing, meaning that there should be no unneccessary spaces in each line. Also, only one line will be left in between each block. As for comments, there must be descriptive comments on top of every block to explain what the follwoing block does. If neccessary, there will be line comments on any complex lines.\\
\\
To conclude, Google's coding style for Python will be used as found on: \url{https://google.github.io/styleguide/pyguide.html}

\section{Project Schedule}

\begin{itemize}

\item \href{run:GanttChart.gan} {Gantt Chart}

\end{itemize}
\section{Project Review}
\begin{itemize}
\item Reflection?
\item What went well?
\item  What did not go well?
\item Modifications to the development plan?
\item Modifications to team meetings, roles and communication? Time management?
\end{itemize}
\end{document}