\documentclass{article}
\usepackage{hyperref}
\usepackage{booktabs}
\hypersetup{colorlinks=true,
    linkcolor=blue,
    citecolor=blue,
    filecolor=blue,
    urlcolor=blue,
    unicode=false}
\usepackage{tabularx}
\usepackage{url}
\usepackage{hyperref}
\hypersetup{colorlinks=true,
    linkcolor=blue,
    citecolor=blue,
    filecolor=blue,
    urlcolor=blue,
    unicode=false}
\title{SE 3XA3: Development Plan\\DNA Says}
\author{Team 10, Team Name: DNA
		\\ Kareem Abdel Mesih (abdelk2)
		\\ John-Paul Dakran (dakranj)
		\\ Shady Nessim (nessimss)
}
\date{}
\begin{document}
\begin{table}[hp]
\caption{Revision History} \label{TblRevisionHistory}
\begin{tabularx}{\textwidth}{llX}
\toprule
\textbf{Date} & \textbf{Developer(s)} & \textbf{Change}\\
\midrule
2016/09/29 & Shady Nessim & Rough Draft\\
2016/09/29 & John-Paul Dakran &  Revision \#1\\
2016/09/29 & Kareem Abdel Mesih &  Revision \#2\\
2016/09/30 & Shady Nessim & Linked Gantt Chart\\
2016/09/30 & All & Final Revision\\
2016/10/10 & John-Paul Dakran & Revision \#3\\
\bottomrule
\end{tabularx}
\end{table}
\newpage
\maketitle
\newpage
\section{Team Meeting Plan} 
The team will meet at a minimum once a week however if the workload for the project increases, team meetings will increase in frequency as agreed upon by the team. The weekly team meeting will occur on Wednesdays at 10:30 AM - 11:20 AM in the ETB Cafe.\\
\\
The duties of the chair will be to take charge in the meeting. This person will break the meeting down into different sections so all items can be covered. They will also make sure the topic of discussion is relevant to the work that needs to be completed. Kareem will chair the meetings.\\
\\
The meetings will begin by discussing the previous weeks workload and activities to decide upon any changes that need to be made and to evaluate group decisions. Once this is complete, the meeting will move towards discussing the current weeks activities. Each group member will express their opinion on the work that needs to be completed. Finally, the work will be distributed evenly among the group members and the meeting will commence.\\
\\
During meetings, a set of notes or "Meeting Minutes" will be recorded to have a record of what occurred during each meeting. This will provide a clear history of what was discussed, decided upon, and a list each persons duties. Group members will alternate recording the meeting minutes - beginning with Kareem, then John-Paul, and finally Shady - then we will repeat.


\section{Team Communication Plan}
The Git Repository will serve as a means for group members and teaching staff to share and view project files. The files that are pushed to Git will be committed with a meaningful message describing the changes that have been made to the specific file. \\
\\
Email and Facebook Messenger will be the platforms that the group members use to communicate with each other. They will serve the purpose discussing ideas, sharing files, and confirming team meetings. Each group member should keep an eye on their email and Facebook messenger account to stay updated with ongoing team activities.\\
\\
All group members will exchange telephone numbers and will be used to contact each other to discuss ideas and plan team meetings. Any questions that need to be answered as soon as possible should be asked through SMS for fastest response time. \\
\\
All issues that occur will be posted on the Issue Tracker. When an issue occurs, all group members will have a discussion about the issue and pitch ideas on how to solve the issue. A meeting will be called if the discussion to solve the issue needs to be discussed further. Once an issue is solved, this will be noted on the Issue Tracker.

\section{Team Member Roles}
The team leader will be Kareem. Kareem will also chair each of the meetings. Group members will alternate recording the meeting minutes - beginning with Kareem, then John-Paul, and finally Shady - then we will repeat.
\\
\begin{itemize}
\item Knowledge and Experience (1-7):
\begin{itemize}
\item Kareem- Documentation: 7, Git: 2, LaTeX: 3, Technology: 7
\item  JP- Documentation: 5, Git: 5, LaTeX: 3, Technology: 6
\item Shady- Documentation: 5, Git: 4, LaTeX: 2, Technology: 5
\item * Technology refers to Python \& Pygame
\end{itemize}
\item Roles: 
\paragraph{Kareem}
\begin{itemize}
\item Record the required soundclips
\item Create the required shapes and buttons
\item Program the logic/back-end
\item Prepare meeting discussions
\end{itemize}
\paragraph{John-Paul}
\begin{itemize}
\item LaTeX preparations and organization
\item Program the user interface/front-end
\end{itemize}
\paragraph{Shady}
\begin{itemize}
\item Git preparations and organization
\item Test the code and maintain it
\end{itemize}
\end{itemize}



\section{Git Workflow Plan}


\begin{itemize}
\item All issues will be posted on the issue tracker.
\item Descriptive commit messages will be used when committing to Git. 
\item Milestones will be set, along with their expected completion date and posted on Git for all group members to be aware of.
\item Labels will be used to prioritize and organize issues, and merge requests when needed.
\end{itemize}
\section{Proof of Concept Demonstration Plan}
\begin{itemize}
\item The implementation of this project should not be difficult, however the process of recording the sounds will be time consuming.
\item Testing should also not be difficult, as strings will be used and compared to test the addition of new moves to the previous pattern.
\item Testing with family and friends will play an integral role in determining the functionality of the program.
\item This code will run on any platform that can run Python.
\item Pygame is required to develop this project. It is simple to install once one knows which versions are compatible with the current version of Python (3.5).
\end{itemize}
\paragraph{Overcoming Risks:}
With regards to compatibility issues, the latest pygame was released in 2009, which fortunately enough is compatible with the current Python version available. However, if Python releases a new update, it might not be compatible anymore. The option to download an older version of Python is still available however, and can be used to continue the project. In fact, there would be no need to update, if one happens to be available.
\section{Technology}
\begin{itemize}
\item Python will be used to develop this project.
\item It will run in its basic IDE Version 3.5.
\item Framework testing will be automated to test the different cases and outcomes of the game. Family and friends will test the overall functionality and performance of the game.
\item LaTeX will be used to generate required documents.
\end{itemize}
\section{Coding Style}
\begin{itemize}
\item Descriptive variable names
\item Descriptive function names
\item Consistency in spacing:
\begin{itemize}
\item One space before and after the operators
\item No spaces on the insides of any brackets
\item One empty line to separate blocks
\item One space before line comments
\item No spaces in between block comment and code blocks
\end{itemize}
\item Comments
\begin{itemize}
\item Descriptive
\item Not too short of a sentence
\item Not everywhere, only when required
\end{itemize}
\end{itemize}
\section{Project Schedule}
\begin{itemize}

\item \href{run:GanttChart.gan} {Gantt Chart}\\

\end{itemize}
\section{Project Review}
\begin{itemize}
\item Reflection?
\item What went well?
\item  What did not go well?
\item Modifications to the development plan?
\item Modifications to team meetings, roles and communication? Time management?
\end{itemize}
\end{document}