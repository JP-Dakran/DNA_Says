\documentclass[12pt, titlepage]{article}
\usepackage{booktabs}
\usepackage{tabularx}
\usepackage{hyperref}
\usepackage{indentfirst}
\hypersetup{
    colorlinks,
    citecolor=black,
    filecolor=black,
    linkcolor=red,
    urlcolor=blue
}
\usepackage[round]{natbib}
\title{SE 3XA3: Test Plan\\DNA Says}
\author{Team \#10, Team Name: DNA
		\\ Kareem Abdel Mesih (abdelk2)
		\\ John-Paul Dakran (dakranj)
		\\ Shady Nessim (nessimss)
}
\date{\today}
\begin{document}
\maketitle
\pagenumbering{roman}
\tableofcontents
\listoftables
\listoffigures
\begin{table}[bp]
\caption{\bf Revision History}
\begin{tabularx}{\textwidth}{p{3cm}p{2cm}X}
\toprule {\bf Date} & {\bf Version} & {\bf Notes}\\
\midrule
Date 1 & 1.0 & Notes\\
Date 2 & 1.1 & Notes\\
\bottomrule
\end{tabularx}
\end{table}
\newpage
\pagenumbering{arabic}
\section{General Information}
\subsection{Purpose}

\par In the engineering process, verification and validation of the requirements outlined in the Software Requirements Specification (SRS) document is essential. This process is executed through a series of tests executed on the requirements to prove that the functionality of the game is correct. This document serves the purpose of outlining how the requirements will be validated and verified.
\\
\par The implementation of the game DNA Says consists of numerous functional capabilities. These functional capabilities range from detecting user input to outputting a correct sound at a precise given time. The complete set of requirements will be broken down into specific and simple tests to prove the functionality of each specific requirement. 


\subsection{Scope}

\par The main objective of this document is to outline an agreed upon set of tests that will be performed on the software system to validate its functionality. The scope of the testing for this game includes testing the animations, sounds, buttons, integration of the system, and all functional and non-functional requirements outline in the Software Requirements Specification (SRS) document.


\subsection{Acronyms, Abbreviations, and Symbols}
	
\begin{table}[hbp]
\caption{\textbf{Table of Abbreviations}} \label{Table}
\begin{tabularx}{\textwidth}{p{3cm}X}
\toprule
\textbf{Abbreviation} & \textbf{Definition} \\
\midrule
SRS & Software Requirement Specification\\
PoC & Proof of Concept\\
GUI & Graphical User Interface\\
\bottomrule
\end{tabularx}
\end{table}


\begin{table}[!htbp]
\caption{\textbf{Table of Definitions}} \label{Table}
\begin{tabularx}{\textwidth}{p{3cm}X}
\toprule
\textbf{Term} & \textbf{Definition}\\
\midrule
Term1 & Definition1\\
Term2 & Definition2\\
\bottomrule
\end{tabularx}
\end{table}	
\subsection{Overview of Document}

\par This document outlines a collection of information about the software system - DNA Says - that is in the process of creation. The test plan document begins by describing the software system and its functionality. It then proceeds to introducing the reader with the test team and the plan for testing - I.e. Testing tools and the testing schedule.
\\
\par Next, the tests for functional and non-functional requirements will be described. Each test will have a type, initial state, input, output, and description of how the test will be performed. The same format will be used for the next section which outlines the tests for the proof of concept.
\\
\par Proceeding, the reader will be introduced to a concise comparison between the original implementation and the implementation that is currently in the process of creation. Next the unit testing plan will be revealed to the reader which describes the unit testing of the internal functions and output files. The test plan document will be concluded by the appendix which will hold a list of symbolic parameters and survey questions for user testing.


\section{Plan}
	
\subsection{Software Description}
\subsection{Test Team}
\subsection{Automated Testing Approach}
\subsection{Testing Tools}
\subsection{Testing Schedule}
		
See Gantt Chart at the following url ...
\section{System Test Description}
	
\subsection{Tests for Functional Requirements}
\subsubsection{Area of Testing1}
		
\paragraph{Title for Test}
\begin{enumerate}
\item{test-id1\\}
Type: Functional, Dynamic, Manual, Static etc.
					
Initial State: 
					
Input: 
					
Output: 
					
How test will be performed: 
					
\item{test-id2\\}
Type: Functional, Dynamic, Manual, Static etc.
					
Initial State: 
					
Input: 
					
Output: 
					
How test will be performed: 
\end{enumerate}
\subsubsection{Area of Testing2}
...
\subsection{Tests for Nonfunctional Requirements}
\subsubsection{Area of Testing1}
		
\paragraph{Title for Test}
\begin{enumerate}
\item{test-id1\\}
Type: 
					
Initial State: 
					
Input/Condition: 
					
Output/Result: 
					
How test will be performed: 
					
\item{test-id2\\}
Type: Functional, Dynamic, Manual, Static etc.
					
Initial State: 
					
Input: 
					
Output: 
					
How test will be performed: 
\end{enumerate}
\subsubsection{Area of Testing2}
...
\section{Tests for Proof of Concept}
\subsection{Area of Testing1}
		
\paragraph{Title for Test}
\begin{enumerate}
\item{test-id1\\}
Type: Functional, Dynamic, Manual, Static etc.
					
Initial State: 
					
Input: 
					
Output: 
					
How test will be performed: 
					
\item{test-id2\\}
Type: Functional, Dynamic, Manual, Static etc.
					
Initial State: 
					
Input: 
					
Output: 
					
How test will be performed: 
\end{enumerate}
\subsection{Area of Testing2}
...
	
\section{Comparison to Existing Implementation}	
				
\section{Unit Testing Plan}
		
\subsection{Unit testing of internal functions}
		
\subsection{Unit testing of output files}		
\bibliographystyle{plainnat}
\bibliography{SRS}
\newpage
\section{Appendix}
This is where you can place additional information.
\subsection{Symbolic Parameters}
The definition of the test cases will call for SYMBOLIC\_CONSTANTS.
Their values are defined in this section for easy maintenance.
\subsection{Usability Survey Questions?}
This is a section that would be appropriate for some teams.
\end{document}