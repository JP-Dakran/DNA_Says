\documentclass{article}

\usepackage{tabularx}
\usepackage{booktabs}

\title{SE 3XA3: Problem Statement\\DNA Says}

\author{Team 10, DNA
		\\ Kareem Abdel Mesih (abdelk2)
		\\ John-Paul Dakran (dakranj)
		\\ Shady Nessim (nessimss)
}

\date{}

\begin{document}

\begin{table}[hp]
\caption{Revision History} \label{TblRevisionHistory}
\begin{tabularx}{\textwidth}{llX}
\toprule
\textbf{Date} & \textbf{Developer(s)} & \textbf{Change}\\
\midrule
14 September 2016 & All & First draft of document\\
21 September 2016 & Kareem Abdel Mesih & Refinement and finalizing\\
28 September 2016 & John-Paul Dakran & Conversion to LaTex\\
\bottomrule
\end{tabularx}
\end{table}

\newpage

\maketitle

\newpage
\section{What problem is being solved?}

\paragraph{Video games have always been one of the top choices with regards to entertainment. They are also named as one of the great ways to help with boredom. This project is simply a remake of the famous game Simon Says, with a little modification that is discussed later in this document. This interactive game is going to serve the purpose of allowing people of all ages, whether bored or simply having a break, to enjoy a fun and amusing game. The main basis of Simon Says is to remember a given pattern, and iterate it back. This project, in addition, could enhance one’s visual memory along with their auditory memory.}

\section{Is this an important problem?}

\paragraph{In a survey that has been conducted by the Washington Post in August 2005, statistics show that fifty-five percent of all employees working in the United States of America are “not engaged” in their work, due to boredom. Along with those employees, everyone else also suffers from boredom at one point everyday. It is an issue that is common amongst all individuals that develops counter productivity and discomfort. It is concluded that boredom is an important, and a serious problem.}

\section{What is the context of the problem in hand?}

\paragraph{For this problem, the current stakeholders are the developers, the individuals who suffer from boredom, along with those who are simply looking for an entertaining game to play. Individuals who are looking to develop their visual or auditory memory are also stakeholders. The basis of this game is to remember a specific pattern that is given, one step at a time, and iterate it back. This means that at level x, there will be x elements in the pattern to repeat. This project is going to include three different modes: Kareem Says, JP Says, and Shady Says. Although all three modes are based on the original game, each person is going to create a slight modification to the game to create some originality, and fun. This project is going to be implemented using Python and Pygame, and could be run on any computer.}

\end{document}

\mh{comment}

\end{document}