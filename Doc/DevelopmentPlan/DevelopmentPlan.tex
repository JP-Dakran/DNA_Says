\documentclass{article}
\usepackage{hyperref}
\usepackage{booktabs}
\hypersetup{colorlinks=true,
    linkcolor=blue,
    citecolor=blue,
    filecolor=blue,
    urlcolor=blue,
    unicode=false}

\usepackage{tabularx}
\title{SE 3XA3: Development Plan\\DNA Says}
\author{Team 10, Team Name: DNA
		\\ Kareem Abdel Mesih (abdelk2)
		\\ John-Paul Dakran (dakranj)
		\\ Shady Nessim (nessimss)
}
\date{}
\begin{document}
\begin{table}[hp]
\caption{Revision History} \label{TblRevisionHistory}
\begin{tabularx}{\textwidth}{llX}
\toprule
\textbf{Date} & \textbf{Developer(s)} & \textbf{Change}\\
\midrule
2016/09/29 & Shady Nessim & Rough Draft\\
2016/09/29 & John-Paul Dakran &  Revision \#1\\
2016/09/29 & Kareem Abdel Mesih &  Revision \#2\\
2016/09/30 & Shady Nessim & Linked Gantt Chart\\
2016/09/30 & All & Final Revision\\
\bottomrule
\end{tabularx}
\end{table}
\newpage
\maketitle
\newpage
\section{Team Meeting Plan}
\begin{itemize}
\item When- Wednesdays at 10:30 AM - 11:20 AM
\item Where- ETB Cafe
\item Frequency- Every week, unless it is unnecessary 
\item Roles:
\paragraph{Kareem}
\begin{itemize}
\item Record the required soundclips
\item Create the required shapes and buttons
\item Program the logic/back-end
\item Prepare meeting discussions
\end{itemize}
\paragraph{John-Paul}
\begin{itemize}
\item LaTeX preparations and organization
\item Program the user interface/front-end
\end{itemize}
\paragraph{Shady}
\begin{itemize}
\item Git preparations and organization
\item Test the code and maintain it
\end{itemize}
\item Rules for Agendas
\begin{itemize}
\item Kareem will chair the meetings.
\item Take turns to record minutes, written statement and homework.
\end{itemize}
\end{itemize}
\section{Team Communication Plan}
\begin{itemize}
\item Git Repository: share project files
\item Email/Facebook Messenger: share other files
\item Phone/SMS: discussions, questions and tips
\item Issue Tracker: issues that come up
\end{itemize}
\section{Team Member Roles}
\begin{itemize}
\item Leader: Kareem
\item Scribe: different person every meeting
\item Knowledge and Experience (1-7):
\begin{itemize}
\item Kareem- Documentation: 7, Git: 2, LaTeX: 3, Technology: 7
\item  JP- Documentation: 5, Git: 5, LaTeX: 3, Technology: 6
\item Shady- Documentation: 5, Git: 4, LaTeX: 2, Technology: 5
\end{itemize}
\item Roles: (as stated before)
\paragraph{Kareem}
\begin{itemize}
\item Record the required soundclips
\item Create the required shapes and buttons
\item Program the logic/back-end
\item Prepare meeting discussions
\end{itemize}
\paragraph{John-Paul}
\begin{itemize}
\item LaTeX preparations and organization
\item Program the user interface/front-end
\end{itemize}
\paragraph{Shady}
\begin{itemize}
\item Git preparations and organization
\item Test the code and maintain it
\end{itemize}
\end{itemize}
\section{Git Workflow Plan}
\begin{itemize}
\item All issues will be posted on the issue tracker.
\item Descriptive commit messages will be used when committing to Git. 
\item Milestones will be set, along with their expected completion date and posted on Git for all group members to be aware of.
\item Labels will be used to prioritize and organize issues, and merge requests when needed.
\end{itemize}
\section{Proof of Concept Demonstration Plan}
\begin{itemize}
\item The implementation of this project should not be difficult, however the process of recording the sounds will be time consuming.
\item Testing should also not be difficult, as strings will be used and compared to test the addition of new moves to the previous pattern.
\item Testing with family and friends will play an integral role in determining the functionality of the program.
\item This code will run on any platform that can run Python.
\item Pygame is required to develop this project. It is simple to install once one knows which versions are compatible with the current version of Python (3.5).
\end{itemize}
\paragraph{Overcoming Risks:}
With regards to compatibility issues, the latest pygame was released in 2009, which fortunately enough is compatible with the current Python version available. However, if Python releases a new update, it might not be compatible anymore. The option to download an older version of Python is still available however, and can be used to continue the project. In fact, there would be no need to update, if one happens to be available.
\section{Technology}
\begin{itemize}
\item Python will be used to develop this project.
\item It will run in its basic IDE Version 3.5.
\item Framework testing will be automated to test the different cases and outcomes of the game. Family and friends will test the overall functionality and performance of the game.
\item LaTeX will be used to generate required documents.
\end{itemize}
\section{Coding Style}
\begin{itemize}
\item Descriptive variable names
\item Descriptive function names
\item Consistency in spacing:
\begin{itemize}
\item One space before and after the operators
\item No spaces on the insides of any brackets
\item One empty line to separate blocks
\item One space before line comments
\item No spaces in between block comment and code blocks
\end{itemize}
\item Comments
\begin{itemize}
\item Descriptive
\item Not too short of a sentence
\item Not everywhere, only when required
\end{itemize}
\end{itemize}
\section{Project Schedule}
\begin{itemize}
\item \href{run:GanttChart.gan} {Gantt Chart}\\
\end{itemize}
\section{Project Review}
\begin{itemize}
\item Reflection?
\item What went well?
\item  What did not go well?
\item Modifications to the development plan?
\item Modifications to team meetings, roles and communication? Time management?
\end{itemize}
\end{document}